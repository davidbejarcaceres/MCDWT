%%% Local Variables:
%%% mode: latex
%%% TeX-master: "<none>"
%%% End:

\title{Motion Compensated Discrete Wavelet Transform (MCDWT)}

\author{Vicente González Ruiz}

\maketitle
\tableofcontents

\section{Intro}

\href{https://en.wikipedia.org/wiki/Video}{Video} data contain high
amounts of redundancy, spatial and temporal. For this reason, most of
video encoders compress the input sequece of
\href{https://en.wikipedia.org/wiki/Digital_image}{images} (which are
matrices of \href{https://en.wikipedia.org/wiki/Pixel}{pixels})
basically in two stages: (1) a transform stage in the
\href{https://en.wikipedia.org/wiki/Time_domain}{time} and in the
\href{https://www.quora.com/What-is-spatial-domain-in-image-processing}{spatial}
domains that produces a collection of spatially
\href{https://en.wikipedia.org/wiki/Decorrelation}{uncorrelated}
\href{https://en.wikipedia.org/wiki/Discrete_wavelet_transform}{coefficients},
and (2) an
\href{https://en.wikipedia.org/wiki/Entropy_encoding}{entropy
  encoding} phase which removes the statistical redundancy that can
still remains after the decorrelation (transform). These coefficients
have two interesting features:
\begin{enumerate}
\item Usually, a \textbf{
  \href{https://vicente-gonzalez-ruiz.github.io/symbol_compression/}{smaller
  entropy}} than the original pixels. This helps to increase
  the \href{https://en.wikipedia.org/wiki/Data_compression_ratio}{compression
  ratio}.
\item The transform domain provides a
  \textbf{\href{https://en.wikipedia.org/wiki/Image_resolution}{multiresolution}
    representation (spatial and temporal)} of the visual information (spatial
  \href{http://inst.eecs.berkeley.edu/~ee290t/sp04/lectures/videowavelet_UCB1-3.pdf}{scalability}).
\end{enumerate}

\section{Distrete Wavelet Transform (DWT)}
\href{https://en.wikipedia.org/wiki/Discrete_wavelet_transform}{(2D-)DWT}
is a digital transform that, applied to an image, performs spatial
decorrelation (2D) and obtains a
\href{https://vicente-gonzalez-ruiz.github.io/image_transformations_for_coding/index.html#x1-3500024}{multiresolution
  dyadic representation} of such image, conforming a collection of DWT
subbands. The forward transform converts the image into a set of
frequency subbands with coefficients representing different spatial
areas and frequency orientations. There is an equivalence between
(DWT) subbands (a \emph{(subband) decomposition}) and
\href{http://fourier.eng.hmc.edu/e161/lectures/canny/node3.html}{Laplacian
  pyramids}, and it is quite simple to
\href{https://vicente-gonzalez-ruiz.github.io/pyramids-and-wavelets/}{pass
  from a decomposition representation to a pyramid representation and
  viceversa}.

\href{https://github.com/vicente-gonzalez-ruiz/MCDWT/blob/master/src/DWT.py}{DWT.py}
implements the forward and backward (2D-)DWT for color images. $L$ and
$H$ stands for \emph{low-pass filtered} and \emph{high-pass filtered},
respectively.  More information about the implementation can be found at
\href{https://pywavelets.readthedocs.io/en/latest/index.html}{PyWavelets}.

\lstinputlisting[firstline=9, lastline=105, language=python, caption=DWT.py]{../src/DWT.py}

The multilevel DWT is a recursive use of the (1-level) DWT applied to
the $LL$ subband. For example, if the DWT is applied $k$-times to the
$LL$ subband, recursively, a decomposition of $k+1$-levels is generated. For
the sake of simplicity, we will denote the subbands $\{LH^k, HL^k,
HH^k\}$ as only $H^k$, and $LL^k$ as only $L^k$.

\subsection{Scalability provided by DWT}
Depending on the number of subbands levels (pyramid levels in the
pyramid representation) that we use, we can reconstruct a image at
different spatial scales (resolutions). For example, if an image $s_1$ has been transformed using the DWT of 2 levels, we get:
\begin{itemize}
\item Subband $s_1.L^2$ with the scale 2.
\item Subband $s_1.L^1=s_1.L=\text{iDWT}(s_1.L^2, S_1.H^2)$ (``iDWT'' =
  ``inverse DWT'' = ``backward DWT'') with the scale 1.
\item $s_1.L^0=\text{iDWT}(s_1.L, s_1.H)$ (the original image) with the scale 0.
\end{itemize}

\subsection{Example}
\begin{verbatim}
# You must be in the 'src' directory.

# 2D 1-levels forward DWT of the first image of the 'stockholm' sequence:
./DWT.py -i ../sequences/stockholm/000 -d /tmp/000

# Visualize the subbands:
display -normalize /tmp/000_LL # <- LL1
display -normalize /tmp/000_LH # <- LH1
display -normalize /tmp/000_HL # <- HL1
display -normalize /tmp/000_HH # <- HH1

# 2D 1-levels forward DWT of the LL subband:
./DWT.py -i /tmp/000_LL
-d /tmp/000_LL

# Visualize the subbands:
display -normalize /tmp/000_LL_LL # <- LL2
display -normalize /tmp/000_LL_LH # <- LH2
display -normalize /tmp/000_LL_HL # <- HL2
display -normalize /tmp/000_LL_HH # <- HH2

# The result of the 2D 2-levels forward DWT are the subbands: LL2,
# LH2, HL2, HH2, LH1, HL1, HH1

# The scale 2 of the image is represented by LL2.

# The scale 1 can be reconstructed using the 2D 1-levels backward DWT
# on the 000_LL_{LL, LH, HL, HH} subbands:
./DWT.py -b -d /tmp/000_LL -i /tmp/scale1 && display -normalize /tmp/scale1

# The reconstruction '/tmp/scale1' and the original scale 1
# '/tmp/000_LL' have the same resolution:
file /tmp/scale1 /tmp/000_LL

# Floating-point arithmetic of DWT introduces minimal distortions:
cmp /tmp/scale1 /tmp/000_LL

# Basically, distortion is white noise:
../tools/show_differences.sh -1 /tmp/scale1 -2 /tmp/000_LL -o /tmp/diffs.png
display -normalize /tmp/diffs.png

# The scale 0 is recovered using again the inverse transform:
cp /tmp/scale1 /tmp/000_LL
./DWT.py -b -d /tmp/000 -i /tmp/scale0 && display -normalize /tmp/scale0
\end{verbatim}

\section{Motion DWT (MDWT)}
DWT can be applied to a sequence of images by simply transforming each
image of the sequence independently. This is done, for example, in the
\href{https://en.wikipedia.org/wiki/JPEG_2000}{motion JPEG2000 video
  compression standard}. Notice that only the spatial redundancy is
exploited. All the temporal redundancy still remains in the video.

\begin{figure}
\centering
\imgw{800}{graphics/forward_MDWT.svg}
\caption{Forward MDWT step.}
\end{figure}

\lstinputlisting[firstline=16, lastline=62, language=python, caption=MDWT.py]{../src/MDWT.py}

\subsection{Scalability provided by MDWT}
MDWT sequences (of (subband) decompositions) are scalable in space and
in time. Spatial scalability is a direct consequence of DWT. On the other
hand, it is trivial to observe that MDWT provides \emph{fully}
temporal scalability (we can access to the images randomly) because
each image of the input sequence is transformed independently.

\subsection{Example}
\begin{verbatim}
# You must be in the 'src' directory.

# Motion 2D 1-levels forward DWT of the 'stockholm' sequence:
python -O ./MDWT.py -i ../sequences/stockholm/ -d /tmp/stockholm_

# Visualize the subbands:
rm /tmp/*.png
for i in /tmp/stockholm_00?_LL; do convert -normalize $i $i.png; done
animate /tmp/*.png
rm /tmp/*.png
for i in /tmp/stockholm_00?_LH; do convert -normalize $i $i.png; done
animate /tmp/*.png
rm /tmp/*.png
for i in /tmp/stockholm_00?_HL; do convert -normalize $i $i.png; done
animate /tmp/*.png
rm /tmp/*.png
for i in /tmp/stockholm_00?_HH; do convert -normalize $i $i.png; done
animate /tmp/*.png

# Motion 2D 1-levels backward DWT:
python -O MDWT.py -b -i /tmp/recons_ -d /tmp/stockholm_

# Visualization of the reconstruction:
rm -f /tmp/*.png; for i in /tmp/recons_*; do convert -normalize $i $i.png; done; animate /tmp/*.png
\end{verbatim}

\section{Video transform alternatives}
To remove both, spatial and temporal redundancies, two different
alternatives are available: (1) t+2D transforms and (2) 2D+t
transforms. In a t+2D transform, the video is first
\href{https://en.wikipedia.org/wiki/Digital_filter\#Analysis_techniques}{analyzed}
(transformed) over the time domain and next, over the space domain. A
2D+t transform does just the opposite. Each choice has a number of
\emph{pros} and \emph{cons}. For example, in a t+2D transform we can
apply directly any image predictor based on
\href{https://en.wikipedia.org/wiki/Motion_estimation}{motion
  estimation} because the input is a normal video. However, if we
implement a 2D+t transform, the input to the motion estimator is a
sequence decompositions.
\href{http://www.polyvalens.com/blog/wavelets/theory}{The overwhelming
  majority of DWT's} are not
\href{http://www.polyvalens.com/blog/wavelets/theory}{shift
  invariant}, which basically means $\text{DWT}(s[t]) \neq
\text{DWT}(s[t+x])$, where $x$ is a displacement of $s[t]$ along one of the
signal domains.  Therefore, motion estimators which compare pixel values
will not work on the (subband) decomposition domain. On the other
hand, if we want to provide true spatial scalability (processing only
those
\href{https://www.tutorialspoint.com/dip/spatial_resolution.htm}{spatial
  resolutions} (scales) necessary to get a spatially scaled of our
video), a t+2D transformed video presents some drawbacks:

\begin{itemize}
\item In order to implement a lossless system, the t stage must be
  carried out at full (scale 0) resolution.
\item We can restrict the perfect reconstruction only to the scale 0
  case, but in this case, a spatial resolution representation of the
  motion information should be available at the decoder. This can be
  done, but in general, scalabilitity decreases the compression ratio
  (on the motion data in this case).
\item We can use the full (scale 0) resolution version of the motion
  information for reconstructing all the scales of the texture
  (interpolating it for all those scales greater that 0), but in this
  case the memory and computing requirements at the decoders will
  increase. Therefore, depending on the resolution of the scale 0 and the
  computational resources, this could be affordable or not.
\item As an alternative to this last problem, we can decrease the
  accuracy of the motion information in order to make it suitable for
  the spatial resolution. However, this will increase the distortion
  of the reconstruction compared to the previous solution.
\end{itemize}

Finally, it is important to realize that the presence of the motion
data in the code-stream introduces also complexity into the decoding
process because, in a quality scalable scenario, it is not trivial
(specially when non-linear systems such as those based on ME are
involved) to decide how to interleave the motion and the texture
information in a code-stream that can be decoded following different
orders, depending on the type of used scalability and the avaliable
bandwidth.

\section{Motion Compensated Discrete Wavelet Transform (MCDWT)}
MCDWT is a 2D+t transform. The 2D stage is MDWT. The t stage is a
1D-DWT, which removes the temporal redundancy between adjacent $H$
subbands by means of
\href{https://en.wikipedia.org/wiki/Motion_compensation}{Motion
  Compensation (MC)}. A special characteristic of MCDWT is that the
motion information is not transmitted from the
\href{https://en.wikipedia.org/wiki/Encoder}{encoder} to the
\href{https://en.wikipedia.org/wiki/Decoder}{decoder}. This is
possible because to estimate de motion at the encoder, only the
information that it is accesible by the decoder is used.

\subsection{MCDWT butterfly}

MCDWT butterly inputs three decompositions $a=\{a.L, a.H\}$, $b=\{b.L,
b.H\}$ and $c=\{c.L, c.H\}$, and outputs a residue subband
$\tilde{b}.H$, which replaces to $b.H$ in the original $b$
decomposition. So, after the use of the bufferfly, we get $a=\{a.L,
a.H\}$ (an
\href{https://en.wikipedia.org/wiki/Video_compression_picture_types}{intra-coded
  ``I'' image}), $\tilde{b}=\{b.L, \tilde{b}.H\}$ (a
\href{https://en.wikipedia.org/wiki/Video_compression_picture_types}{bidirectionally
  predicted ``B'' image}) and $c=\{c.L, c.H\}$ (another intra-coded
``I'' image). This replacement is fully reversible because the forward
transform will use only the information that the backward transform
has access to. Notice that the
\href{http://www.vtvt.ece.vt.edu/research/references/video/DCT_Video_Compression/Zhang92a.pdf}{pyramid
  domain} (which is invariant to the pixels displacements) has been
used to estimate and compensate the $H$ subbands.

\begin{figure}
\centering
\imgw{1200}{graphics/forward_butterfly.svg}
\caption{Forward MCDWT butterfly.}
\end{figure}
\lstinputlisting[firstline=20, lastline=67, language=python, caption={MCDWT.py (extract)}]{../src/MCDWT.py}

Notice that, even if $a$ and $c$ (at full resolution) are available at
the decoder, only $b.L$ is. So, it does not make sense to search the
low resolution structures of $b.L$ in the high resolution images $a$
and $c$, because even if a perfect match is achieved (think about the
three decompositions $a$, $b$ and $c$ are identical), the prediction error
would be different than $0$ as a consequence of that the high
resolution information is missing in $b.L$.

\subsection{MCDWT forward transform}

The MCDWT forward transform is the result of appliying the MCDWT
butterfly to the full input sequence of decompositions with different
distances between them. As a result, except for the first and the last
image of the sequence, all the $H$-subbands are temporally
decorrelated.

\begin{figure}
\centering
\imgw{1200}{graphics/temporal_decorrelation.svg}
\caption{MCDWT forward transform.\label{fig:forward_MCDWT}}
\end{figure}
\lstinputlisting[firstline=69, lastline=138, language=python, caption={MCDWT.py (extract)}]{../src/MCDWT.py}

\subsection{(Spatial) Multiresolution}

Spatial dyadic multiresolution can be obtained by appliying a sequence of MDWT+MCDWT steps (see Fig.~\ref{fig:multiresolution}).

\begin{figure}
\centering
\imgw{1200}{graphics/multiresolution.svg}
\caption{MDWT+MCDWT multiresolution procedure. The input to the second
  iteration of the MDWT+MCDWT transform is the output of a previous
  (first) iteration MDWT+MCDWT transform. The second iteration is only
  applied to the scale 1.\label{fig:multiresolution}}
\end{figure}


\subsection{Scalability provided by MCDWT}
The decomposition generated by the MCDWT is identical to
MDWT. Therefore, MCDWT provides the same spatial scalability than
MDWT. In order to obtain several spatial resolutions, MCDWT should be
applied to the low-frequency subbands, recursively. As an example, in
the Figure~\ref{fig:multiresolution}, MCDWT has been applied to the
output of the example of the Figure~\ref{fig:forward_MCDWT}. Notice
that between two consecutive iterations of MCDWT, MDWT must be applied
to the low-frequency subband in order to reduce the resolution of the
analysis.

On the other hand, MCDWT decreases the temporal scalability offered by
MDWT to allowing only dyadic access (depending on required image and
the number of MCDWT levels, more than one decomposition needs to be
inversely transformed). For example, if only one step of the backward
transform is applied to the example of the
Figure~\ref{fig:forward_MCDWT}, only the decompositions of $s_0$,
$s_2$ and $s_4$ will be reconstructed (the second
\href{https://en.wikipedia.org/wiki/Temporal_resolution}{temporal
  resolution}). If a second iteration of the backward transform is
used, all the decompositions are rendered.

\subsection{Example}
\begin{verbatim}
# You must be in the 'src' directory.

# Motion 2D 1-levels forward DWT of the 'stockholm' sequence:
python -O ./MDWT.py -i ../sequences/stockholm/ -d /tmp/stockholm_

# Motion Compensated 1D 1-levels forward DWT:
python -O MCDWT.py -d /tmp/stockholm_ -m /tmp/mc_stockholm_

# First and last decompositions are not transformed by MCDWT:
ls -l /tmp/stockholm_000_HH /tmp/mc_stockholm_000_HH
ls -l /tmp/stockholm_004_HL /tmp/mc_stockholm_004_HL

# The rest, are transformed:
ls -l /tmp/stockholm_001_LH /tmp/mc_stockholm_001_LH
ls -l /tmp/stockholm_002_HL /tmp/mc_stockholm_002_HL
ls -l /tmp/stockholm_003_HH /tmp/mc_stockholm_003_HH

# Showing statistics:
python ../tools/show_statistics.py -i /tmp/stockholm_001_LH
python ../tools/show_statistics.py -i /tmp/mc_stockholm_001_LH
python ../tools/show_statistics.py -i /tmp/stockholm_001_HL
python ../tools/show_statistics.py -i /tmp/mc_stockholm_001_HL
python ../tools/show_statistics.py -i /tmp/stockholm_001_HH
python ../tools/show_statistics.py -i /tmp/mc_stockholm_001_HH

# Visualization of the H-subbands
rm -f /tmp/*.png; for i in /tmp/mc_stockholm_00?_LH; do convert -normalize $i $i.png; done; animate /tmp/*.png
rm -f /tmp/*.png; for i in /tmp/mc_stockholm_00?_HL; do convert -normalize $i $i.png; done; animate /tmp/*.png
rm -f /tmp/*.png; for i in /tmp/mc_stockholm_00?_HH; do convert -normalize $i $i.png; done; animate /tmp/*.png

# Motion Compensated 1D 1-levels backward DWT:
python -O ./MCDWT.py -b -m /tmp/mc_stockholm_ -d /tmp/recons_MCDWT_

# Motion 2D 1-levels backward DWT:
python -O ./MDWT.py -b -d /tmp/recons_MCDWT_ -i /tmp/recons_MDWT_

# Visualization of the reconstruction:
rm -f /tmp/*.png; for i in /tmp/recons_MDWT_*; do ../tools/sub_32768_128.py $i $i.png; done; animate /tmp/recons_MDWT_*.png

# Visualize the original sequence:
rm -f /tmp/stockholm/*.png; cp -r ../sequences/stockholm /tmp; for i in /tmp/stockholm/*; do ../tools/sub_32768_128.py $i $i.png; done; animate /tmp/stockholm/*.png
\end{verbatim}

\begin{comment}

\section{Video \href{http://inst.eecs.berkeley.edu/~ee290t/sp04/lectures/videowavelet_UCB1-3.pdf}{scalability}}
MCDWT inputs a \href{https://en.wikipedia.org/wiki/Video}{video} (a
sequence of images) and outputs a video (a sequece of pyramids), in
such a way that when only a portion of the data of the transformed
video is used, a video with a lower
\href{https://en.wikipedia.org/wiki/Temporal_resolution}{temporal
resolution}, lower
\href{https://en.wikipedia.org/wiki/Image_resolution\#Spatial_resolution}{spatial
resolution} or/and lower
\href{https://en.wikipedia.org/wiki/Compression_artifact}{quality} can
be generated.

If all the transformed data is used, then the original video is
restored (MCDWT is a lossless transform). The forward transform's
output has exactly the same number of pyramids than images are in the
input video (for example, no extra motion fields are produced). At
this time, we will focuse only on spatial
\href{http://inst.eecs.berkeley.edu/~ee290t/sp04/lectures/videowavelet_UCB1-3.pdf}{scalability}. Window
Of Interest (WOI) and quality scabilities will be addressed later.

\end{comment}
