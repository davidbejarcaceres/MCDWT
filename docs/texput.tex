%%% Local Variables:
%%% mode: latex
%%% TeX-master: "<none>"
%%% End:

\title{Motion Compensated Discrete Wavelet Transform (MCDWT)}

\author{Vicente González Ruiz}

\maketitle
\tableofcontents

\section{Intro}

\href{https://en.wikipedia.org/wiki/Video}{Video} data contain high
amounts of redundancy, spatial and temporal. For this reason, most of
video encoders compress the input sequece of
\href{https://en.wikipedia.org/wiki/Digital_image}{images} (which are
matrices of \href{https://en.wikipedia.org/wiki/Pixel}{pixels})
basically in two stages: (1) a transform stage in the
\href{https://en.wikipedia.org/wiki/Time_domain}{time} and in the
\href{https://www.quora.com/What-is-spatial-domain-in-image-processing}{spatial}
domains that produces a collection of spatially
\href{https://en.wikipedia.org/wiki/Decorrelation}{uncorrelated}
\href{https://en.wikipedia.org/wiki/Discrete_wavelet_transform}{coefficients},
and (2) an
\href{https://en.wikipedia.org/wiki/Entropy_encoding}{entropy
  encoding} phase which removes the statistical redundancy that can
still remains after the decorrelation (transform). These coefficients
have two interesting features:
\begin{enumerate}
\item Usually, a \textbf{
  \href{https://vicente-gonzalez-ruiz.github.io/symbol_compression/}{smaller
  entropy}} than the original pixels. This helps to increase
  the \href{https://en.wikipedia.org/wiki/Data_compression_ratio}{compression
  ratio}.
\item The transform domain provides a
  \textbf{\href{https://en.wikipedia.org/wiki/Image_resolution}{multiresolution}
    representation (spatial and temporal)} of the visual information (spatial
  \href{http://inst.eecs.berkeley.edu/~ee290t/sp04/lectures/videowavelet_UCB1-3.pdf}{scalability}).
\end{enumerate}

\section{Distrete Wavelet Transform (DWT)}
\href{https://en.wikipedia.org/wiki/Discrete_wavelet_transform}{(2D-)DWT}
is a digital transform that, applied to an image, performs spatial
decorrelation (2D) and obtains a
\href{https://vicente-gonzalez-ruiz.github.io/image_transformations_for_coding/index.html#x1-3500024}{multiresolution
  dyadic representation} of such image, conforming a collection of DWT
subbands. The forward transform converts the image into a set of
frequency subbands with coefficients representing different spatial
areas and frequency orientations. There is an equivalence between
(DWT) subbands (a \emph{(subband) decomposition}) and
\href{http://fourier.eng.hmc.edu/e161/lectures/canny/node3.html}{Laplacian
  pyramids}, and it is quite simple to
\href{https://vicente-gonzalez-ruiz.github.io/pyramids-and-wavelets/}{pass
  from a decomposition representation to a pyramid representation and
  viceversa}.

\href{https://github.com/vicente-gonzalez-ruiz/MCDWT/blob/master/src/DWT.py}{DWT.py}
implements the forward and backward (2D-)DWT for color images. $L$ and
$H$ stands for \emph{low-pass filtered} and \emph{high-pass filtered},
respectively.  More information about the implementation can be found at
\href{https://pywavelets.readthedocs.io/en/latest/index.html}{PyWavelets}.

\lstinputlisting[firstline=9, lastline=105, language=python, caption=DWT.py]{../src/DWT.py}

The multilevel DWT is a recursive use of the (1-level) DWT applied to
the $LL$ subband. For example, if the DWT is applied $k$-times to the
$LL$ subband, recursively, a decomposition of $k+1$-levels is generated. For
the sake of simplicity, we will denote the subbands $\{LH^k, HL^k,
HH^k\}$ as only $H^k$, and $LL^k$ as only $L^k$.

\subsection{Scalability provided by DWT}
Depending on the number of subbands levels (pyramid levels in the
pyramid representation) that we use, we can reconstruct a image at
different spatial scales (resolutions). For example, if an image $s_1$ has been transformed using the DWT of 2 levels, we get:
\begin{itemize}
\item Subband $s_1.L^2$ with the scale 2.
\item Subband $s_1.L^1=s_1.L=\text{iDWT}(s_1.L^2, S_1.H^2)$ (``iDWT'' =
  ``inverse DWT'' = ``backward DWT'') with the scale 1.
\item $s_1.L^0=\text{iDWT}(s_1.L, s_1.H)$ (the original image) with the scale 0.
\end{itemize}

\subsection{Example}
\begin{verbatim}
# You must be in the 'src' directory.

# Create the output directory for the 1st-level decomposition:
rm -rf /tmp/000
mkdir /tmp/000

# 2D 1-levels forward DWT of the first image of the 'stockholm' sequence:
python3 ./DWT.py -i ../sequences/stockholm/000 -d /tmp/000/


# Visualize the subbands:
tree /tmp/000
display -normalize /tmp/000/LL.png # <- LL1
display -normalize /tmp/000/LH.png # <- LH1
display -normalize /tmp/000/HL.png # <- HL1
display -normalize /tmp/000/HH.png # <- HH1

# Create the output directory for the 2nd-level decomposition:
mkdir /tmp/000/LL

# 2D 1-levels forward DWT of the LL subband:
./DWT.py -i /tmp/000/LL -d /tmp/000/LL/

# Visualize the subbands:
tree /tmp/0
display -normalize /tmp/000/LL/LL.png # <- LL2
display -normalize /tmp/000/LL/LH.png # <- LH2
display -normalize /tmp/000/LL/HL.png # <- HL2
display -normalize /tmp/000/LL/HH.png # <- HH2

# The result of the 2D 2-levels forward DWT are the subbands: LL2,
# LH2, HL2, HH2, LH1, HL1, HH1

# The scale 2 of the image is represented by LL2.

# The scale 1 can be reconstructed using the 2D 1-levels backward DWT
# on the 000_LL_{LL, LH, HL, HH} subbands:
python3 ./DWT.py -b -d /tmp/000/LL/ -i /tmp/scale1 && display -normalize /tmp/scale1.png

# The reconstruction '/tmp/scale1' and the original scale 1
# '/tmp/000_LL' have the same resolution:
file /tmp/scale1.png /tmp/000/LL.png

# Floating-point arithmetic of DWT introduces minimal distortions:
cmp /tmp/scale1.png /tmp/000/LL.png

# Basically, distortion is white noise:
../tools/show_differences.sh -1 /tmp/scale1.png -2 /tmp/000/LL.png -o /tmp/diffs.png
display -normalize /tmp/diffs.png

# The scale 0 is recovered using again the inverse transform:
cp -f /tmp/scale1.png /tmp/000/LL.png
python3 ./DWT.py -b -d /tmp/000/ -i /tmp/scale0 && display -normalize /tmp/scale0.png
\end{verbatim}

\section{Motion DWT (MDWT)}
DWT can be applied to a sequence of images by simply transforming each
image of the sequence independently. This is done, for example, in the
\href{https://en.wikipedia.org/wiki/JPEG_2000}{motion JPEG2000 video
  compression standard}. Notice that only the spatial redundancy is
exploited. All the temporal redundancy still remains in the video.

\begin{figure}
\centering
\imgw{800}{graphics/forward_MDWT.svg}
\caption{Forward MDWT step.}
\end{figure}

\lstinputlisting[firstline=16, lastline=62, language=python, caption=MDWT.py]{../src/MDWT.py}

\subsection{Scalability provided by MDWT}
MDWT sequences (of (subband) decompositions) are scalable in space and
in time. Spatial scalability is a direct consequence of DWT. On the other
hand, it is trivial to observe that MDWT provides \emph{fully}
temporal scalability (we can access to the images randomly) because
each image of the input sequence is transformed independently.

\subsection{Example}
\begin{verbatim}
# You must be in the 'src' directory.

# Create the output directory for the 1st-level decompositions:
rm -rf /tmp/S
mkdir /tmp/S

# Motion 2D 1-levels forward DWT of the 'stockholm' sequence:
python3 -O ./MDWT.py -i ../sequences/stockholm/ -d /tmp/S/

# Visualize the subbands:
for i in /tmp/S/00?/LL.png; do convert -normalize $i $i.norm; done
animate /tmp/S/00?/LL*.norm
for i in /tmp/S/00?/LH.png; do convert -normalize $i $i.norm; done
animate /tmp/S/00?/LH*.norm
for i in /tmp/S/00?/HL.png; do convert -normalize $i $i.norm; done
animate /tmp/S/00?/HL*.norm
for i in /tmp/S/00?/HH.png; do convert -normalize $i $i.norm; done
animate /tmp/S/00?/HH*.norm

# Motion 2D 1-levels backward DWT:
rm -rf /tmp/s
mkdir /tmp/s
python3 -O MDWT.py -b -i /tmp/s/ -d /tmp/S/

# Visualization of the reconstruction:
for i in /tmp/s/00?.png; do convert -normalize $i $i.norm; done; animate /tmp/s/*.norm
\end{verbatim}

\section{Video transform alternatives}
To remove both, spatial and temporal redundancies, two different
alternatives are available: (1) t+2D transforms and (2) 2D+t
transforms. In a t+2D transform, the video is first
\href{https://en.wikipedia.org/wiki/Digital_filter\#Analysis_techniques}{analyzed}
(transformed) over the time domain and next, over the space domain. A
2D+t transform does just the opposite. Each choice has a number of
\emph{pros} and \emph{cons}. For example, in a t+2D transform we can
apply directly any image predictor based on
\href{https://en.wikipedia.org/wiki/Motion_estimation}{motion
  estimation} because the input is a normal video. However, if we
implement a 2D+t transform, the input to the motion estimator is a
sequence decompositions.
\href{http://www.polyvalens.com/blog/wavelets/theory}{The overwhelming
  majority of DWT's} are not
\href{http://www.polyvalens.com/blog/wavelets/theory}{shift
  invariant}, which basically means $\text{DWT}(s[t]) \neq
\text{DWT}(s[t+x])$, where $x$ is a displacement of $s[t]$ along one of the
signal domains.  Therefore, motion estimators which compare pixel values
will not work on the (subband) decomposition domain. On the other
hand, if we want to provide true spatial scalability (processing only
those
\href{https://www.tutorialspoint.com/dip/spatial_resolution.htm}{spatial
  resolutions} (scales) necessary to get a spatially scaled of our
video), a t+2D transformed video presents some drawbacks:

\begin{itemize}
\item In order to implement a lossless system, the t stage must be
  carried out at full (scale 0) resolution.
\item We can restrict the perfect reconstruction only to the scale 0
  case, but in this case, a spatial resolution representation of the
  motion information should be available at the decoder. This can be
  done, but in general, scalabilitity decreases the compression ratio
  (on the motion data in this case).
\item We can use the full (scale 0) resolution version of the motion
  information for reconstructing all the scales of the texture
  (interpolating it for all those scales greater that 0), but in this
  case the memory and computing requirements at the decoders will
  increase. Therefore, depending on the resolution of the scale 0 and the
  computational resources, this could be affordable or not.
\item As an alternative to this last problem, we can decrease the
  accuracy of the motion information in order to make it suitable for
  the spatial resolution. However, this will increase the distortion
  of the reconstruction compared to the previous solution.
\end{itemize}

Finally, it is important to realize that the presence of the motion
data in the code-stream introduces also complexity into the decoding
process because, in a quality scalable scenario, it is not trivial
(specially when non-linear systems such as those based on ME are
involved) to decide how to interleave the motion and the texture
information in a code-stream that can be decoded following different
orders, depending on the type of used scalability and the avaliable
bandwidth.

\section{Motion Compensated Discrete Wavelet Transform (MCDWT)}
MCDWT is a 2D+t transform. The 2D stage is MDWT. The t stage is a
1D-DWT, which removes the temporal redundancy between adjacent $H$
subbands by means of
\href{https://en.wikipedia.org/wiki/Motion_compensation}{Motion
  Compensation (MC)}. A special characteristic of MCDWT is that the
motion information is not transmitted from the
\href{https://en.wikipedia.org/wiki/Encoder}{encoder} to the
\href{https://en.wikipedia.org/wiki/Decoder}{decoder}. This is
possible because to estimate de motion at the encoder, only the
information that it is accesible by the decoder is used.

\subsection{MCDWT butterfly}

MCDWT butterly inputs three decompositions $a=\{a.L, a.H\}$, $b=\{b.L,
b.H\}$ and $c=\{c.L, c.H\}$, and outputs a residue subband
$\tilde{b}.H$, which replaces to $b.H$ in the original $b$
decomposition. So, after the use of the bufferfly, we get $a=\{a.L,
a.H\}$ (an
\href{https://en.wikipedia.org/wiki/Video_compression_picture_types}{intra-coded
  ``I'' image}), $\tilde{b}=\{b.L, \tilde{b}.H\}$ (a
\href{https://en.wikipedia.org/wiki/Video_compression_picture_types}{bidirectionally
  predicted ``B'' image}) and $c=\{c.L, c.H\}$ (another intra-coded
``I'' image). This replacement is fully reversible because the forward
transform will use only the information that the backward transform
has access to. Notice that the
\href{http://www.vtvt.ece.vt.edu/research/references/video/DCT_Video_Compression/Zhang92a.pdf}{pyramid
  domain} (which is invariant to the pixels displacements) has been
used to estimate and compensate the $H$ subbands.

\begin{figure}
\centering
\imgw{1200}{graphics/forward_butterfly.svg}
\caption{Forward MCDWT butterfly.\label{fig:forward_butterfly}}
\end{figure}
\lstinputlisting[firstline=24, lastline=57, language=python, caption={MCDWT.py (extract)}]{../src/MCDWT.py}
\lstinputlisting[firstline=4, lastline=9, language=python, caption={simple\_average.py (extract)}]{../src/simple_average.py}

Notice that, even if $a$ and $c$ (at full resolution) are available at
the decoder, only $b.L$ is. So, it does not make sense to search the
low resolution structures of $b.L$ in the high resolution images $a$
and $c$, because even if a perfect match is achieved (think about the
three decompositions $a$, $b$ and $c$ are identical), the prediction error
would be different than $0$ as a consequence of that the high
resolution information is missing in $b.L$.

\subsection{MCDWT forward transform}

The MCDWT forward transform is the result of appliying the MCDWT
butterfly to the full input sequence of decompositions with different
distances between them. As a result, except for the first and the last
image of the sequence, all the $H$-subbands are temporally
decorrelated.

\begin{figure}
\centering
\imgw{1200}{graphics/temporal_decorrelation.svg}
\caption{MCDWT forward transform.\label{fig:forward_MCDWT}}
\end{figure}
\lstinputlisting[firstline=69, lastline=138, language=python, caption={MCDWT.py (extract)}]{../src/MCDWT.py}

\subsection{(Spatial) Multiresolution}

Spatial dyadic multiresolution can be obtained by appliying a sequence of MDWT+MCDWT steps (see Fig.~\ref{fig:multiresolution}).

\begin{figure}
\centering
\imgw{1200}{graphics/multiresolution.svg}
\caption{MDWT+MCDWT multiresolution procedure. The input to the second
  iteration of the MDWT+MCDWT transform is the output of a previous
  (first) iteration MDWT+MCDWT transform. The second iteration is only
  applied to the scale 1.\label{fig:multiresolution}}
\end{figure}


\subsection{Scalability provided by MCDWT}
The decomposition generated by the MCDWT is identical to
MDWT. Therefore, MCDWT provides the same spatial scalability than
MDWT. In order to obtain several spatial resolutions, MCDWT should be
applied to the low-frequency subbands, recursively. As an example, in
the Figure~\ref{fig:multiresolution}, MCDWT has been applied to the
output of the example of the Figure~\ref{fig:forward_MCDWT}. Notice
that between two consecutive iterations of MCDWT, MDWT must be applied
to the low-frequency subband in order to reduce the resolution of the
analysis.

On the other hand, MCDWT decreases the temporal scalability offered by
MDWT to allowing only dyadic access (depending on required image and
the number of MCDWT levels, more than one decomposition needs to be
inversely transformed). For example, if only one step of the backward
transform is applied to the example of the
Figure~\ref{fig:forward_MCDWT}, only the decompositions of $s_0$,
$s_2$ and $s_4$ will be reconstructed (the second
\href{https://en.wikipedia.org/wiki/Temporal_resolution}{temporal
  resolution}). If a second iteration of the backward transform is
used, all the decompositions are rendered.

\subsection{Example}
\begin{verbatim}
# You must be in the 'src' directory.

# Motion 2D 1-levels forward DWT of the 'stockholm' sequence:
python -O ./MDWT.py -i ../sequences/stockholm/ -d /tmp/stockholm_

# Motion Compensated 1D 1-levels forward DWT:
python -O MCDWT.py -d /tmp/stockholm_ -m /tmp/mc_stockholm_

# First and last decompositions are not transformed by MCDWT:
ls -l /tmp/stockholm_000_HH /tmp/mc_stockholm_000_HH
ls -l /tmp/stockholm_004_HL /tmp/mc_stockholm_004_HL

# The rest, are transformed:
ls -l /tmp/stockholm_001_LH /tmp/mc_stockholm_001_LH
ls -l /tmp/stockholm_002_HL /tmp/mc_stockholm_002_HL
ls -l /tmp/stockholm_003_HH /tmp/mc_stockholm_003_HH

# Showing statistics:
python ../tools/show_statistics.py -i /tmp/stockholm_001_LH
python ../tools/show_statistics.py -i /tmp/mc_stockholm_001_LH
python ../tools/show_statistics.py -i /tmp/stockholm_001_HL
python ../tools/show_statistics.py -i /tmp/mc_stockholm_001_HL
python ../tools/show_statistics.py -i /tmp/stockholm_001_HH
python ../tools/show_statistics.py -i /tmp/mc_stockholm_001_HH

# Visualization of the H-subbands
rm -f /tmp/*.png; for i in /tmp/mc_stockholm_00?_LH; do convert -normalize $i $i.png; done; animate /tmp/*.png
rm -f /tmp/*.png; for i in /tmp/mc_stockholm_00?_HL; do convert -normalize $i $i.png; done; animate /tmp/*.png
rm -f /tmp/*.png; for i in /tmp/mc_stockholm_00?_HH; do convert -normalize $i $i.png; done; animate /tmp/*.png

# Motion Compensated 1D 1-levels backward DWT:
python -O ./MCDWT.py -b -m /tmp/mc_stockholm_ -d /tmp/recons_MCDWT_

# Motion 2D 1-levels backward DWT:
python -O ./MDWT.py -b -d /tmp/recons_MCDWT_ -i /tmp/recons_MDWT_

# Visualization of the reconstruction:
rm -f /tmp/*.png; for i in /tmp/recons_MDWT_*; do python ../tools/substract_offset.py -i $i -o $i.png; done; animate /tmp/recons_MDWT_*.png

# Visualize the original sequence:
rm -f /tmp/stockholm/*.png; cp -r ../sequences/stockholm /tmp; for i in /tmp/stockholm/*; do python ../tools/substract_offset.py -i $i -o $i.png; done; animate /tmp/stockholm/*.png

# Let's clear the motion compensated subbands and reconstruct the sequence ...

# Create a zero subband:
bash ../tools/create_black_image.sh; python ../tools/add_offset.py -i /tmp/zero.png -o /tmp/zero16.png -f 32768
python ../tools/show_statistics.py -i /tmp/zero16.png

# "Delete" the motion compensated subbands:
cp /tmp/zero16.png /tmp/mc_stockholm_001_LH
cp /tmp/zero16.png /tmp/mc_stockholm_001_HL
cp /tmp/zero16.png /tmp/mc_stockholm_001_HH
cp /tmp/zero16.png /tmp/mc_stockholm_002_LH
cp /tmp/zero16.png /tmp/mc_stockholm_002_HL
cp /tmp/zero16.png /tmp/mc_stockholm_002_HH
cp /tmp/zero16.png /tmp/mc_stockholm_003_LH
cp /tmp/zero16.png /tmp/mc_stockholm_003_HL
cp /tmp/zero16.png /tmp/mc_stockholm_003_HH

# Motion Compensated 1D 1-levels backward DWT:
python -O ./MCDWT.py -b -m /tmp/mc_stockholm_ -d /tmp/recons2_MCDWT_

# Motion 2D 1-levels backward DWT:
python -O ./MDWT.py -b -d /tmp/recons2_MCDWT_ -i /tmp/recons2_MDWT_

# Visualization of the reconstruction:
rm -f tmp/recons2_MDWT_*.png; for i in /tmp/recons2_MDWT_*; do python ../tools/substract_offset.py -i $i -o $i.png; done; animate /tmp/recons2_MDWT_*.png

# Visualization of the differences:
rm -f /tmp/diff*.png; for i in {0..4}; do ii=$(printf "%03d" $i); bash ../tools/show_differences.sh -1 /tmp/recons_MDWT_$ii -2 /tmp/recons2_MDWT_$ii -o /tmp/diff_$i.png; done; animate /tmp/diff*.png
\end{verbatim}

\section{Adaptive motion compensation based on the distortion of the prediction error}
As can be seen in the MCDWT bufferfly (see
Fig.~\ref{fig:forward_butterfly}), both predictions $[b_a.H]$ and
$[b_c.H]$ have the same weight to build the prediction
\begin{equation}
  [\hat{b.H}] = \frac{[b_a.H] + [b_c.H]}{2}.
\end{equation}
This simple computation, that can have very effective for example in
\href{https://biteable.com/blog/tips/video-transitions-effects-examples/}{dissolves}
video transitions, can also be counterproductive when objects appear
only in one of the reference images. For this reason, in this section
a different predictor is proposed, based on the prediction error
generated during the ME process.

Let $[b_a.L]$ the prediction generated for the subband $[b.L]$ using
$[a.L]$ as reference and $[a.L]\rightarrow [b.L]$ as motion, and let
$[b_c.L]$ the prediction generated for $[b.L]$ using $[c.L]$ as
reference and $[c.L]\rightarrow [b.L]$ as motion. We now compute the
prediction errors
\begin{equation}
  \begin{array}{l}
    {[e_a.L]} = [b.L] - [b_a.L]\\
    {[e_c.L]} = [b.L] - [b_c.L].
  \end{array}
\end{equation}

Now, we define the similarity matrices as
\begin{equation}
  \begin{array}{l}
    {[s_a.L]} = \frac{1}{1+{|[e_a.L]|}}\\
    {[s_c.L]} = \frac{1}{1+{|[e_c.L]|}}.    
  \end{array}
  \label{eq:weighted_prediction}
\end{equation}
Notice that, if (for example) the error $[e_a.L]_{x,y}=0$, the
similarity is $[s_a.L]_{x,y}=1$ (the maximum similarity), and if the
error is high, the similarity tends to be $0$.

With this information, that can be recovered by the decoder, the
improved prediction is defined as
\begin{equation}
  [\hat{b.H}] = \frac{[b_a.H][s_a.L]+[s_c.L][b_c.H]}{[s_a.L]+[s_c.L]}.
\end{equation}

Notice that, if (for example) $[s_a.L]_{x,y}=1$ and
$[s_c.L]_{x,y}\approx 0$, then
$[\hat{b.H}]_{x,y}\approx [b_a.H]_{x,y}$, and viceversa. If
$[s_a.L]_{x,y}=[s_c.L]_{x,y}$, then
$[\hat{b.H}]_{x,y}=\frac{[b_a.H]+[b_c.H]}{2}$ (even if both similarities are small).

\lstinputlisting[firstline=4, lastline=15, language=python, caption={weighted\_average.py (extract)}]{../src/weighted_average.py}

\section{Orthogonal, orthonormal, and biorthogonal transforms}
In signal processing\footnote{In mathemathics the definition of
  orthogonality refers to characteristics such as the basis of the
  transform forms an orthognal espace, where is is impossible to
  represent one of the basis as a linear combination of the rest of
  basis (or in other words, if the inner product is zero).}, a
transform (such as the Discrete Cosine Transform, the Walsh-Hadamard
Transform or the Karhunen-Loeve Transform) is orthogonal when the
coefficients generated by the transform are uncorrelated (there is no
way to infer one coefficient from another).

If the norm of all the basis of the an orthogonal transform is one,
then the transform is said orthonormal. Orthonormal transforms are
interesting because:
\begin{enumerate}
\item They are energy preserving: the energy of the output is the same
  than the energy of the input. This means that, for example, a
  quantization error produced in a coefficient of the transform will
  generate the same quantization error at the output (the complete
  signal) of the inverse transform. The same holds by the forward
  transform.
\item The transform matrix of the inverse transform is the transpose
  of the forward transform. In orthogonal transforms, the transform
  matrix of the inverse transform is the inverse of the transform
  matrix of the forward transform.
\end{enumerate}

Biorthogonal transforms do no satisfy any of these features: they are
not energy preserving (this can be also observed because the
frequency-domain responses of the analysis and synthesis filters are
not symmetric), and there is not an algebraic way (matrix
transposition/inversion) to compute the backward transform from the
forward one, and viceversa. This, that can be considered as a
drawback, gives an extra degree of freedom to design the analysis and
the synthesis filters (whose only requirement is that transform pair
to be reversible), providing in general the possibility of using more
sofisticated filters such as those based on non-linear filtering, as
for example, those based on motion estimation algorithms.

In general, each subband $b$ of a decomposition generated by a
biorthogonal 2D-DWT transform have a different subband gain
$\alpha_b$. Usually, the lower the frequency of the subband, the
higher the gain. Notice also, that these gains also are different for
each different transform.

Subband gains are important in lossy signal compression because they
quantify the relative importance of the wavelet coefficients of the
different subbands when we introduce distortion in the wavelet
domain. Thus, for example, if we decide to quantize a wavelet
coefficient, the amount of distortion that we are generating in the
signal domain will depend on the subband where that coefficient is
localized. In general, low-frequency coefficients are more
``important'' that high-frequency ones.

To compute the subband gains we have two options:
\begin{enumerate}
\item The algebraic way. We will need the expressions of the four
  filters (two anaylsis filters and two synthesis filters) and deduce
  the gains.
\item The algorithmic way. We will need to compute the energy of the
  impulse response of the inverse transform, when we apply such
  impulse to each one of the subbands of the decomposition. Supposing
  that, after appliying the DWT to an image, the coefficients of the
  subband HH are the least energetic with an energy $x$, the subband
  gain for subband $b$ is computed as
  \begin{equation}
    \alpha_b = \epsilon_b/x,
  \end{equation}
  where $\epsilon_b$ is the energy of the reconstruction when the
  impulse signal is localized at $b$. Notice that all the gains should
  be larger than one.
\end{enumerate}

This computation can be also applied to MCDWT, by computing the subband
gains as a function of the number $T$ of temporal decompositions.

\section{Encoding of B-type subbands}
The subband $\tilde{b.H}$ is no longer needed for applying the
butterfly to different decompositions and scales. For this reason, we
can compress this subband. As a result, after using the butterfly to
all the decompositions of the sequence and all the scales, all the $H$
subbands will be compressed, excepting the intra-coded image of each
GOP. Notice that the redundancy exploited in B-type $H$ subbands is
temporal.

To compress the $\tilde{b.H}$ subbands, the following algorithm can be
used:
\begin{enumerate}
\item [1.] Compute the prediction error for the $[b.L]$ subband
\begin{equation}
  [\tilde{b.L}] = [b.L] - [\hat{b.L}]
\end{equation}
where
\begin{equation}
  [\hat{b.L}] = \frac{[b_a.L][s_a.L]+[s_c.L][b_c.L]}{[s_a.L]+[s_c.L]}.
\end{equation}

\item [2.] Compute the 2D-DWT of subband $L$
  \begin{equation}
    \tilde{b}=\text{DWT}([\tilde{b.L}]).
  \end{equation}

\item [3.] Find the indices for $\tilde{b.L}$ that sorts it in descending order by energy
  \begin{equation}
    \text{indices}=\text{unravel\_index}(\text{argsort}(\text{abs}(\tilde{b.L}),
    b.L.\text{width}, b.L.\text{height})).
  \end{equation}
%\item [3.] For each coefficient $x,y$ in $\tilde{b}.L$:
%  \begin{enumerate}
%  \item [a.] $d_{i=x\times b.\text{width}+y} = (\text{abs}(\tilde{b}_{x,y}), x, y)$
%  \end{enumerate}
%\item [4.] Sort $d$ in desceding order sorted by field 0 /* use the amplitude of luminance */.
\item Go over $\tilde{b}.L$ sending the wavelet coefficients of subbands
  $LH^2$, $HL^2$ and $HH^2$ in descending order by energy
%\item [5.] Estimate the nearest power of two smaller or equan than the maximum amplitude
%  \begin{equation}
%    \lambda_0 = \lambda = 1 << \text{int}(\log_2(\text{abs}(d_{0}))).
%  \end{equation}
%\item [6.] For each coefficient $i$ in $d$:
%  \begin{enumerate}
%  \item [a.] $x,y = d_i[1:2]$.
%  \item [b.] If any of
%    $1<<\text{int}(\log_2(\tilde{b.H}_{x,y}))==\lambda_0$ then send
%    $\text{sign}(\tilde{b.H}_{x,y})$ for the three (LH, HL and HH)
%    subbands.
%  \item [b.] Send $(\hat{b.H}_{x,y}~\text{bitwise-and}~\lambda)/\lambda$ for
%    the three (LH, HL and HH) subbands.
  \end{enumerate}
\item [7.] $\lambda >>= 1$.
\item [8.] Goto step 6.
\end{enumerate}

\section{Encoding of I-type subbands}
After the the $T$-levels MCDWT transform of a sequence of
decompositions, $T+1$ temporal subbands are generated, and the
temporal subband $L^{T+1}$ has a one $H$ I-type subband for each
GOP. Notice that the redundancy exploited in the I-type $H$ subbands
is spatial.

To compress the $a.H$ subband in the last iteration of the butterfly,
the following algorithm can be used:
\begin{enumerate}
\item [1.] Compute the 2D-DWT of the $a.L$ subband
\begin{equation}
  LL^2, LH^2, HL^2, HH^2 = \text{DWT}(LL). 
\end{equation}

\item [2.] Desplace 2 bits to the left the $H$ coefficients to create
  space for encoding the subband
  \begin{equation}
    \begin{array}{l}
      LH^2 <<= 2 \\
      HL^2 <<= 2 \\
      HH^2 <<= 2.
    \end{array}
  \end{equation}
\item [3.] Encode de subband:
  \begin{equation}
    \begin{array}{l}
      HL^2 += 1 \\
      HH^2 += 2.
    \end{array}
  \end{equation}
\item [4.] Create an array with the 3 matrices
  \begin{equation}
    H = [LH^2, HL^2, HH^2].
  \end{equation}
\item [5.] Create a linear array with the 3 flattened matrices
  \begin{equation}
    H'=\text{ravel}(H).
  \end{equation}
\item [6.] Find the indices for the $LH^2$, $HL^2$ and $HH^2$ matrices
  that sort $H'$ in descending order by energy
  ($\text{width}=LH^2.\text{shape}[0]$ and $\text{height}=LH^2.\text{shape}[1]$)
  \begin{equation}
    \text{indices} =
    \text{unravel\_index}(\text{argsort}(\text{abs}(H')), \text{width},
    \text{height}).
  \end{equation}
\item [7.] Go over $H'$ sending the wavelet coefficients of subbands
  $LH^2$, $HL^2$ and $HH^2$ in descending order by energy
  \begin{equation}
    \begin{array}{l}
      \text{for~}d\text{~in range}(3): \\
      ~ \text{for~}y\text{~in range}(\text{height}): \\
      ~~ \text{for~}x\text{~in range}(\text{width}): \\
      ~~~ i = 3\times(\text{height}\times \text{width})+y\times \text{width} + x \\
      ~~~ \text{send}(H[H'[\text{indice}[i]] \text{~mod~} 4][H'[\text{indice}[i]] >> 2]).
    \end{array}
  \end{equation}
\end{enumerate}

\begin{comment}

\section{Progressive reconstruction}
The forward bufferfly should reduce the energy of subband $b.H$$ after
substracting a prediction (see Eq.~\ref{eq:weighted_prediction}) which
is generated using the subbands $a.H$ and $c.H$ as references. If the
prediction is good, the energy of $\tilde{b.H}$ will be small and
viceversa. If the prediction is good, the quantization of the subbands
$a.H$, $b.H$ and $c.H$ should leave more energy in the subbands $a.H$
and $c.H$ than in $b.H$. Therefore, if the prediction is good, the
progressive reconstruction of the subbands (using a progressively
small quantization step) should reconstruct first the information of
$a.H$ and $c.H$.

\section{Coefficients encoding}
Subband $\tilde{b.H}$ is not used any more as a reference and
therefore, it can be compressed. The compression algorithm sort the
coefficients of $\tilde{b.H}$ by the amplitude of the coefficients of
$\tilde{b.L}$, considering that the prediction error will affect in
the same way the low and the high frequencies of $b$. Then, the sorted
coefficients are DPCM encoded, quantized, and entropy encoded.

\section{Multilevel MCLT}
The distance between images $a$, $b$ and $c$ should decrease the
efficiency of the predictions, generating more energy in the
low-frequency subbands. Therefore, the quantization when $T>1$ should
transmit more information of the low-frequency subbands than of the
high-frequency ones when the subbands are quantized.

\end{comment}
